% status: 0
% chapter: TBD

\title{Synthetic Data Generation}

\author{Tim Whitson}
\affiliation{%
  \institution{Indiana University}
  \streetaddress{Smith Research Center}
  \city{Bloomington}
  \state{IN}
  \postcode{47408}
  \country{USA}
}
\email{tdwhitso@indiana.edu}

% The default list of authors is too long for headers}
\renewcommand{\shortauthors}{T. Whitson}

\begin{abstract}
Synthetic data solves the problems of a lack of data or a lack of access
to data. In this paper, we explore synthetic data -- why it is necessary,
how to generate it, and how to use it.
\end{abstract}

\keywords{hid-sp18-526, machine learning, synthetic data}

\maketitle

\section{Introduction}

Data analysis has changed the landscape of business and research. All data
has a story, and data scientists are tasked with telling that story. However,
what if the data is inaccessible to the data scientists? Are they out of
luck or is it possible to create data that is still actionable?

The solution to this problem is synthetic data. Synthetic data can be generated
to look like real data. In the same way that predictions can be made using
existing data, synthetic data can be generated. While it may seem illogical,
synthetic data can also be analyzed and used to train models.

The most notable synthetic data generator is the Synthetic Data
Vault. Developed at MIT, the Synthetic Data Vault uses machine learning
techniques to model relational tables. The models can then be used to generate
entirely synthetic tables which are true to the form of the originals.\cite{}

\section{Generation}

The first step in synthetic data analysis is to generate the data. The
Synthetic Data Vault accomplishes this task using machine learning. First,
the natural data is analyzed and modeled. Then, the models are used to
generate the synthetic data. The benefit of machine learning is that the
models are able to recreate realistic synthetic data points.

Take, for example, a database filled with hospital patient data. Due to
the private nature of health data, analysts might not have access to the
data. However, a synthetic recreation of the data is possible, which would
be perfectly useable for analysis. Suppose there are fields for "height" and
"weight". These features have a relationship, positive correlation, which
would be captured by the machine learning algorithm.\cite{} Also suppose
height is normally distributed. Combine all of these factors together and a
normal distribution of heights could be generated with matching weights. While
this example is oversimplified, it shows that synthetically-generated data
is far more than just random.

\bibliographystyle{ACM-Reference-Format}
\bibliography{report}

