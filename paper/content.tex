% status: 0
% chapter: TBD

\title{Synthetic Data Vault}

\author{Tim Whitson}
\affiliation{%
  \institution{Indiana University}
  \streetaddress{Smith Research Center}
  \city{Bloomington}
  \state{IN}
  \postcode{47408}
  \country{USA}
}
\email{tdwhitso@indiana.edu}

% The default list of authors is too long for headers}
\renewcommand{\shortauthors}{T. Whitson}

\begin{abstract}
Synthetic data solves the problems of a lack of data or a lack of access to
data. Synthetic Data Vault (SDV), which was developed in 2016 at MIT, is the
premier technology in synthetic data generation. Here, we explore SDV ---
why it is necessary, how it generates data, and how to use the synthetic
data. %FIXME: improve
\end{abstract}

\keywords{hid-sp18-526, machine learning, synthetic data}

\maketitle

\section{Introduction}

Data analysis has changed the landscape of business and research. All data
has a story, and data scientists are tasked with telling that story. However,
what if the data is inaccessible to the data scientists? Are they out of
luck or is it possible to create data that is still actionable?

The solution to this problem is synthetic data. Synthetic data can be generated
to look like real data. In the same way that predictions can be made using
existing data, synthetic data can be generated. While it may seem illogical,
synthetic data can also be analyzed and used to train models.

The most notable synthetic data generator is the Synthetic Data Vault
(SDV). Developed at MIT, the Synthetic Data Vault uses machine learning
techniques to model relational tables. The models can then be used to generate
entirely synthetic tables which are true to the form of the originals.\cite{}
%FIXME: citation

\section{Why Synthetic Data?}

Data science and analysis has driven industry to new areas never before
explored. One look at Kaggle or DrivenData shows the usefulness and value
that data provides organizations who nourish it. However, organizations
also face hurdles with their data (or lack thereof). Consider healthcare
companies or defense contractors with highly sensitive data that cannot
be shared publicly. Also consider small businesses who don't possess the
knowledge or manpower to collect real data.

Assuming that synthetic data is as insightful as real data, these organizations
would all benefit from synthetic data. In the seminal work for SDV, Neha
Patki outlines two motivations for using synthetic data: ``populating
sample databases'' and ``Scaling Data Science Efforts''\cite{}. Consider,
for example, a company using a Kaggle competition to gain insight into their
data. The company cannot release real data on their users due to privacy
concerns. This is one example amongst many where synthetic data solves the
issue of data availablity (or scaled efforts, as more data scientists can
now access the data). %FIXME: citation incomplete

\subsection{Privacy}

Whether for business reasons or privacy issues, most data is sensitive. Any
attempt at a public display of data could lead to a loss of revenue, lawsuits,
damage to national security, etc. Synthetic data sticks to the structure
of the original data without any real-world underpinning. So patient data,
for example, could be generated and used by businesses and researchers to
improve patient outcome without any of the pitfalls or added steps of using
real patient data. With the modern climate, user privacy is taken very
seriously and SDV provides a solution to keeping data private. %FIXME: citation

\section{Generating Synthetic Data}

The first step in synthetic data analysis is to generate the data. The SDV
accomplishes this task using machine learning. First, the natural data is
analyzed and modeled. Then, the models are used to generate the synthetic
data. The benefit of machine learning is that the models are able to recreate
realistic synthetic data points.

Most data that would be synthetically modeled sits in a database. Therefore,
SDV works specifically with relational tables inside databases. By analyzing
each column, or feature, of the data, SDV can accurately generate unique data
that conforms to the original. Each numerical row, for example, has a min,
max, mean, etc. which will continue to be followed.

Take, for example, a database filled with hospital patient data. Due to
the private nature of health data, analysts might not have access to the
data. However, a synthetic recreation of the data is possible, which would be
perfectly useable for analysis. Suppose there are fields for ``height'' and
``weight''. These features have a relationship, positive correlation, which
would be captured by the machine learning algorithm. Also suppose height is
normally distributed. Combine all of these factors together and a normal
distribution of heights could be generated with matching weights. While
this example is oversimplified, it shows that synthetically-generated data
is far more than just random data.

\subsection{Process}

First, the user must give a type for each column. ID columns, for example,
are crucial to the generation of new data because they often times are used
as foreign keys in other tables. Any other formatting will also need to be
supplied, such as datetimes. Custom formatting can be supplied via regex.

Next, SDV ``learns'' the database. The tables, and relationship between the
tables, is modeled. With this step, SDV can accurately recreate not only
the information within the database, but its structure as well.

Finally, the data is synthesized. Tables can be synthesized and their child
(dependent) tables can be synthesized as well. Here, the user also has the
option of viewing the model of the database.

%FIXME: image: model pg. 33 from paper

\section{Use of Synthetic Data}

The use of synthetic data goes beyond privacy concerns. As Patki stated,
one use of the SDV is for ``Populating Sample Databases''\cite{}. Data
scientists can generate as much, or as little, data as is necessary. More
data can be generated than currently exists in the original database to
further test scaling.

%FIXME: citation incomplete

The flexibility of SDV is one of its greatest advantages. Instead of requiring
storage of or access to an entire database, all data scientists need are the
models. From those models, they can generate whatever they need to work with.

\subsection{Accuracy}

Synthetic data would not be very useful if it could not take the place of real
data during analysis. In the sample study behind the project, Patki gave both
real and synethesized data to data scientists to run analysis on. The results,
albeit with a small sample, show no statistically significant difference
between real and synthesized data in predictive modeling. Therefore, it can
be tentatively concluded that synthetic data can be used for analysis in
lieu of real data. \cite{}

%FIXME: citation incomplete

\section{Conclusion}

\bibliographystyle{ACM-Reference-Format}
\bibliography{report}

