\section{Apache Accumulo}
\index{Apache Accumulo}
\index{HDFS}

Based on Google's BigTable design, Apache has their own data store called
Accumulo\cite{hid-sp18-526-www-apache-accumulo}. Accumulo overlays the
Hadoop Distributed File System (HDFS) and Apache Zookeeper. Originally
created by the US National Security Agency, Accumulo has a large focus on
security and access control. Every key-value pair in Accumulo has its own
user restrictions. Accumulo is used mostly in other open source projects
and in other Apache projects such as Fluo, Gora, Hive, Pig, and Rya.

Accumulo is a distributed storage system for data, which is simpler than a
typical key-value pair system. Each record in Accumulo has the following
properties: ``Key'', ``Value'', ``Row ID'', ``Column'', ``Timestamp'',
``Family'', ``Qualifier'', and ``Visibility''. The records are stored across
many machines, with Accumolo keeping track of the properties. A monitor is
also provided for information on the current states of the system. A garbage
collector, tablet server (table partition manager), and tracer (for timing)
are also included as well as iterators for data management.

